\documentclass{article}
\usepackage[vmargin={0.75in}, hmargin={0.25in}]{geometry}
\usepackage{amsmath,amssymb,amsfonts}
\usepackage{algorithmic}
\usepackage{graphicx}
\usepackage{textcomp}
\usepackage{xcolor}
\usepackage{array}
\usepackage{ragged2e}
\usepackage{float} % Appending the [H] option forces the placement of a figure in the place it's in the code
\usepackage{multirow, makecell}
\usepackage{longtable}

% \code{} command is used for inline code
\usepackage{listings}
\usepackage{xparse}
\NewDocumentCommand{\code}{v}{%
\texttt{\textcolor{black}{#1}}%
}

% Used for a left-aligned table cell with given width
\newcolumntype{P}[1]{>{\RaggedRight\hspace{0pt}}p{#1}}
% Used for thick column lines in a table
\newcolumntype{?}{!{\vrule width 4\arrayrulewidth}}

\begin{document}

\begin{center}
\LARGE{\textbf{CompactRISC16 (CR16) Instruction Set Architecture (ISA)}}\\[7pt]
\normalsize{Computer Design Laboratory ECE 3710}\\
\normalsize{Fall 2021}\\
\normalsize{The University of Utah}
\end{center}

\large{Table 1: Assembly Instructions and Machine Encodings}
\centering
\footnotesize
\renewcommand{\arraystretch}{1.4}
\begin{longtable}{ | l | l | P{1.5in} ? l | l | l | l ? P{1.37in} | }
\cline{6-7}
\multicolumn{5}{c|}{} & \textbf{ImmHi/} & \textbf{ImmLo/} \\
\cline{4-5}
\multicolumn{3}{c|}{} & \textbf{Opcode} & \textbf{Rdest} & \textbf{Opcode Ext} & \textbf{Rsrc} & \multicolumn{1}{c}{} \\
\hline
\textbf{Mnemonic} & \textbf{Operands} & \textbf{Function} & \textbf{[15:12]} & \textbf{[11:8]} & \textbf{[7:4]} & \textbf{[3:0]} & \textbf{Notes} \\
\Xcline{1-8}{4\arrayrulewidth}
ADD & Rsrc, Rdest & \code{Rdest = Rdest + Rsrc} & 0000 & Rdest & 0000 & Rsrc & \\ \hline
ADDI & Imm, Rdest & \code{Rdest = Rdest + Imm} & 0001 & Rdest & ImmHi & ImmLo & Sign extended Imm \\ \hline
ADD & Rsrc, Rdest & \code{Rdest = Rdest + Rsrc} & 0000 & Rdest & 0000 & Rsrc & \\ \hline
ADDI & Imm, Rdest & \code{Rdest = Rdest + Imm} & 0001 & Rdest & ImmHi & ImmLo & Sign extended Imm \\ \hline
ADDU & Rsrc, Rdest & \code{Rdest = Rdest + Rsrc} & 0000 & Rdest & 0001 & Rsrc & \\ \hline
ADDUI & Imm, Rdest & \code{Rdest = Rdest + Imm} & 0010 & Rdest & ImmHi & ImmLo & Sign extended Imm \\ \hline
ADDC & Rsrc, Rdest & \code{Rdest = Rdest + Rsrc + 1} & 0000 & Rdest & 0010 & Rsrc & \\ \hline
ADDCI & Imm, Rdest & \code{Rdest = Rdest + Imm + 1} & 0011 & Rdest & ImmHi & ImmLo & Sign extended Imm \\ \hline
ADDCUI & Imm, Rdest & \code{Rdest = Rdest + Imm + 1} & 0100 & Rdest & ImmHi & ImmLo & Sign extended Imm \\ \hline
MUL & Rsrc, Rdest & \code{Rdest = Rdest * Rsrc} & 0000 & Rdest & 0011 & Rsrc & \\ \hline
MULI & Imm, Rdest & \code{Rdest = Rdest * Imm} & 0101 & Rdest & ImmHi & ImmLo & Sign extended Imm \\ \hline
SUB & Rsrc, Rdest & \code{Rdest = Rdest - Rsrc} & 0000 & Rdest & 0100 & Rsrc & \\ \hline
SUBI & Imm, Rdest & \code{Rdest = Rdest - Imm} & 0110 & Rdest & ImmHi & ImmLo & Sign extended Imm \\ \hline
CMP & Rsrc, Rdest & \code{Rdest - Rsrc} & 0000 & Rdest & 0101 & Rsrc & \\ \hline
CMPI & Imm, Rdest & \code{Rdest - Imm} & 0111 & Rdest & ImmHi & ImmLo & Sign extended Imm \\ \hline
AND & Rsrc, Rdest & \code{Rdest = Rdest & Rsrc} & 0000 & Rdest & 0110 & Rsrc & \\ \hline
ANDI & Imm, Rdest & \code{Rdest = Rdest & Imm} & 1000 & Rdest & ImmHi & ImmLo & Zero extended Imm \\ \hline
OR & Rsrc, Rdest & \code{Rdest = Rdest | Rsrc} & 0000 & Rdest & 0111 & Rsrc & NOP instruction is OR R0, R0 \\ \hline
ORI & Imm, Rdest & \code{Rdest = Rdest | Imm} & 1001 & Rdest & ImmHi & ImmLo & Zero extended Imm \\ \hline
XOR & Rsrc, Rdest & \code{Rdest = Rdest ^ Rsrc} & 0000 & Rdest & 1000 & Rsrc & \\ \hline
XORI & Imm, Rdest & \code{Rdest = Rdest ^ Imm} & 1010 & Rdest & ImmHi & ImmLo & Zero extended Imm \\ \hline
LSH & Ramount, Rdest & \code{Rdest = Rdest << Ramount} & 0000 & Rdest & 1001 & Ramount & 0 $\le$ Ramount $\le$ 15 since registers are only 16-bits \\ \hline
LSHI & Imm, Rdest & \code{Rdest = Rdest << Imm} & 0000 & Rdest & 1010 & ImmLo & 0 $\le$ ImmLo $\le$ 15 \\ \hline
RSH & Ramount, Rdest & \code{Rdest = Rdest >> Ramount} & 0000 & Rdest & 1011 & Ramount & 0 $\le$ Ramount $\le$ 15 \\ \hline
RSHI & Imm, Rdest & \code{Rdest = Rdest >> Imm} & 0000 & Rdest & 1100 & ImmLo & 0 $\le$ ImmLo $\le$ 15 \\ \hline
ALSH & Ramount, Rdest & \code{Rdest = Rdest <<< Ramount} & 0000 & Rdest & 1101 & Ramount & 0 $\le$ Ramount $\le$ 15 \\ \hline
ALSHI & Imm, Rdest & \code{Rdest = Rdest <<< Imm} & 0000 & Rdest & 1110 & ImmLo & 0 $\le$ ImmLo $\le$ 15 \\ \hline
ARSH & Ramount, Rdest & \code{Rdest = Rdest >>> Ramount} & 0000 & Rdest & 1111 & Ramount & 0 $\le$ Ramount $\le$ 15 \\ \hline
ARSHI & Imm, Rdest & \code{Rdest = Rdest >>> Imm} & 1111 & Rdest & 0000 & ImmLo & 0 $\le$ ImmLo $\le$ 15 \\ \hline
MOV & Rsrc, Rdest & \code{Rdest = Rsrc} & 1111 & Rdest & 0001 & Rsrc & Copies Rsrc into Rdest \\ \hline
MOVIL & Lower Imm, Rdest & \code{Rdest[7:0] = Imm} & 1011 & Rdest & ImmHi & ImmLo & Zero extended Imm, moves immediate value into lower bits of Rdest \\ \hline
MOVIU & Upper Imm, Rdest & \code{Rdest[15:8] = Imm} & 1100 & Rdest & ImmHi & ImmLo & Zero padded Imm, moves immediate value into upper bits of Rdest \\ \hline
B[condition] & Displacement Imm & \code{Relative jump by Imm if [condition]} & 1101 & condition & ImmHi & ImmLo & Immediate is used as a 2's complement program counter/address displacement. [condition] bit patterns are in Table 2. \\ \hline
J[condition] & Rtarget & \code{Absolute jump to Rtarget if [condition]} & 1111 & condition & 0010 & Rtarget & [condition] bit patterns are in Table 2. \\ \hline
JAL & Rlink, Rtarget & \code{Jump to Rtarget, Rlink = PC + 1} & 1111 & Rlink & 0011 & Rtarget & Stores the address of the next instruction in Rlink and jumps to Rtarget, used for subroutines \\ \hline
LPC & Rdest & \code{Rdest = PC} & 1111 & Rdest & 0100 & xxxx & Sets Rdest to the current instruction address/PC \\ \hline
LSF & Rdest & \code{Rdest[4:0] = status flags} & 1111 & Rdest & 0101 & xxxx & Sets 5 least significant bits of Rdest to the current status flags \\ \hline
SSF & Rsrc & \code{Status flags = Rsrc[4:0]} & 1111 & xxxx & 0110 & Rsrc & Sets current status flags to 5 least significant bits of Rsrc \\ \hline
LOAD & Raddr, Rdest & \code{Rdest = Main memory value at Raddr} & 1111 & Raddr & 0111 & Rdest & Used to load data at Raddr into Rdest from main memory \\ \hline
STORE & Rsrc, Raddr & \code{Main memory value at Raddr = Rsrc} & 1111 & Raddr & 1000 & Rsrc & Used to store data at Raddr from Rsrc to main memory \\ \hline
LOADX & Raddr, Rdest & \code{Rdest = External memory at Raddr} & 1111 & Raddr & 1001 & Rdest & Used to load data at Raddr into Rdest from external/peripheral memory/registers \\ \hline
STOREX & Rsrc, Raddr & \code{External memory value at Raddr = Rsrc} & 1111 & Raddr & 1010 & Rsrc & Used to store data at Raddr from Rsrc to external/peripheral memory/registers \\ \hline
NOP &  & \code{No Operation} &  &  &  &  & Alias for: OR R0, R0 \\ \hline
\end{longtable}

\clearpage

\large{Table 2: Bit Patterns of Conditions for B[condition] and J[condition]}
\centering
\footnotesize
\renewcommand{\arraystretch}{1.4}
\begin{longtable}{ | l | l | l | l | }
\hline
\textbf{Mnemonic} & \textbf{Bit Pattern} & \textbf{Description} & \textbf{Status Flags} \\ \Xcline{1-4}{4\arrayrulewidth}
EQ & 0000 & Equal & \code{Z=1} \\ \hline
NE & 0001 & Not Equal & \code{Z=0} \\ \hline
CS & 0010 & Carry Set & \code{C=1} \\ \hline
CC & 0011 & Carry Clear & \code{C=0} \\ \hline
FS & 0100 & Flag Set & \code{F=1} \\ \hline
FC & 0101 & Flag Clear & \code{F=0} \\ \hline
LT & 0110 & Less Than & \code{N=0 and Z=0} \\ \hline
LE & 0111 & Less than or Equal & \code{N=0} \\ \hline
LO & 1000 & Lower than & \code{L=0 and Z=0} \\ \hline
LS & 1001 & Lower than or Same as & \code{L=0} \\ \hline
GT & 1010 & Greater Than & \code{N=1} \\ \hline
GE & 1011 & Greater than or Equal & \code{N=1 or Z=1} \\ \hline
HI & 1100 & Higher than & \code{L=1} \\ \hline
HS & 1101 & Higher than or Same as & \code{L=1 or Z=1} \\ \hline
UC & 1110 & Unconditional & \code{N/A} \\ \hline
& 1111 & Never Jump & \code{N/A} \\ \hline
\end{longtable}

\large{Table 3: Register Conventions}
\centering
\footnotesize
\renewcommand{\arraystretch}{1.4}
\begin{longtable}{ | l | P{2in} | }
\hline
\textbf{Register Index} & \textbf{Convention} \\ \Xcline{1-2}{4\arrayrulewidth}
\code{4'd15} & Stack Pointer with an address starting at \code{0xFFFF} $(2^{16})$ and grows downward towards dynamically allocated memory \\ \hline
\code{4'd14} & Rlink \\ \hline
\code{4'd13} & Return value of subroutine \\ \hline
\code{4'd12} & 1st subroutine argument \\ \hline
\code{4'd11} & 2nd subroutine argument \\ \hline
\code{4'd10} & 3rd subroutine argument \\ \hline
\code{4'd9} & Caller-owned \\ \hline
\code{4'd8} & Caller-owned \\ \hline
\code{4'd7} & Caller-owned \\ \hline
\code{4'd6} & Caller-owned \\ \hline
\code{4'd5} & Callee-owned \\ \hline
\code{4'd4} & Callee-owned \\ \hline
\code{4'd3} & Callee-owned \\ \hline
\code{4'd2} & Callee-owned \\ \hline
\code{4'd1} & Callee-owned \\ \hline
\code{4'd0} & Callee-owned \\ \hline
\end{longtable}

\end{document}
