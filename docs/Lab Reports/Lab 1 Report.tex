\documentclass[conference]{IEEEtran}
\usepackage{cite}
\usepackage{amsmath,amssymb,amsfonts}
\usepackage{algorithmic}
\usepackage{graphicx}
\usepackage{textcomp}
\usepackage{xcolor}
\usepackage{array}
\usepackage{ragged2e}
\def\BibTeX{{\rm B\kern-.05em{\sc i\kern-.025em b}\kern-.08em
    T\kern-.1667em\lower.7ex\hbox{E}\kern-.125emX}}

% Used for a left-aligned table cell with given width
\newcolumntype{P}[1]{>{\RaggedRight\hspace{0pt}}p{#1}}

\begin{document}

\title{Lab 1 Report --- ALU\\
\Large{Computer Design Laboratory ECE 3710}\\
\Large{Fall 2021}\\
\Large{The University of Utah}}

\author{\IEEEauthorblockN{Jacob Peterson}
\IEEEauthorblockA{\textit{Computer Engineering 2022}\\
\textit{University of Utah}\\
Salt Lake City, UT}
\and
\IEEEauthorblockN{Brady Hartog}
\IEEEauthorblockA{\textit{Computer Engineering 2022}\\
\textit{University of Utah}\\
Salt Lake City, UT}
\and
\IEEEauthorblockN{Isabella Gilman}
\IEEEauthorblockA{\textit{Computer Engineering 2023}\\
\textit{University of Utah}\\
Salt Lake City, UT}
\and
\IEEEauthorblockN{Nate Hansen}
\IEEEauthorblockA{\textit{Computer Engineering 2023}\\
\textit{University of Utah}\\
Salt Lake City, UT}
}

\maketitle

\begin{abstract}
This document contains data tables of the ALU (Algorithmic Logic Unit) implementation for Lab 1.
\end{abstract}

\section{Table of ALU Opcodes}
\begin{center}
\begin{tabular}{ | P{5em} | P{5em} | P{12em} | } 
\hline
\textbf{ALU Opcode} & \textbf{Instruction Encoding} & \textbf{Description} \\
\hline
ADD & 0 & Signed addition \\
\hline
ADDU & 1 & Unsigned addition \\
\hline
ADDC & 2 & Signed addition with carry \\
\hline
ADDCU & 3 & Unsigned addition with carry \\
\hline
SUB & 4 & Signed subtraction \\
\hline
SUBU & 5 & Unsigned subtraction \\
\hline
AND & 6 & Bitwise AND \\
\hline
OR & 7 & Bitwise OR \\
\hline
XOR & 8 & Bitwise XOR \\
\hline
NOT & 9 & Bitwise NOT \\
\hline
LSH & 10 & Logical left shift \\
\hline
RSH & 11 & Logical right shift \\
\hline
ALSH & 12 & Arithmetic (sign-extending) left shift \\
\hline
ARSH & 13 & Arithmetic (sign-extending) right shift \\
\hline
\end{tabular}
\end{center}

\vspace{0.2in}
\section{Table of ALU Status Bit Mappings}
\begin{center}
\begin{tabular}{ | P{5em} | P{5em} | P{12em} | } 
\hline
\textbf{Status Bit} & \textbf{Status One-Hot Encoding} & \textbf{Description} \\
\hline
CARRY & \verb|5'b00001| & MSB carry out for unsigned addition \\
\hline
LOW & \verb|5'b00010| & \verb|B < A| for unsigned subtraction \\
\hline
FLAG & \verb|5'b00100| & MSB carry out for signed addition \\
\hline
ZERO & \verb|5'b01000| & Set when \verb|C == 0| \\
\hline
NEGATIVE & \verb|5'b10000| & \verb|B < A| for signed subtraction \\
\hline
\end{tabular}
\end{center}
\end{document}

